\documentclass[10pt,landscape]{article}
\usepackage[utf8]{inputenc}
\usepackage[ngerman]{babel}
\usepackage{tikz}
\usetikzlibrary{shapes,positioning,arrows,fit,calc,graphs,graphs.standard}
\usepackage[nosf]{kpfonts}
\usepackage[t1]{sourcesanspro}
%\usepackage[lf]{MyriadPro}
%\usepackage[lf,minionint]{MinionPro}
\usepackage{multicol}
\usepackage{wrapfig}
\usepackage[absolute,overlay]{textpos}
\usepackage[top=4mm,bottom=7mm,left=5mm,right=5mm]{geometry}
\usepackage[shortlabels]{enumitem}
\usepackage[framemethod=tikz]{mdframed}
\usepackage{microtype}
\usepackage{amsmath}
\usepackage{mathtools}
\usepackage{dirtytalk}
\usepackage{listings}
\usepackage{hyperref}
\usepackage{soul}
\usepackage{comment}


\usepackage{graphicx}

\let\bar\overline
\newcommand{\Lim}[1]{\raisebox{0.5ex}{\scalebox{0.8}{$\displaystyle \lim_{#1}\;$}}}

\DeclarePairedDelimiter\abs{\lvert}{\rvert}%
%\setlist[enumerate,1]{leftmargin=5mm}
\setlist[itemize,1]{leftmargin=5mm}
\setlist{nosep}
\setuldepth{hi}


\definecolor{myblue}{cmyk}{1,.72,0,.38}
\definecolor{mygrey}{RGB}{54,71,76}
\definecolor{vestablue}{HTML}{2B3F48}
\definecolor{pastelred}{HTML}{fbb4ae}
\definecolor{pastelblue}{HTML}{b3cde3}
\definecolor{pastelgreen}{HTML}{ccebc5}
\definecolor{pastelpurple}{HTML}{decbe4}
\definecolor{pastelorange}{HTML}{fed9a6}
\definecolor{pastelyellow}{HTML}{ffffcc}
\definecolor{pastelbrown}{HTML}{e5d8bd}
\definecolor{pastelpink}{HTML}{fddaec}
\definecolor{pastelgrey}{HTML}{f2f2f2}
\definecolor{forestgreen}{HTML}{228B22}



\tikzset{filled/.style={fill=circle area, draw=circle edge, thick},
    outline/.style={draw=circle edge, thick}}

\pgfdeclarelayer{background}
\pgfsetlayers{background,main}

\everymath\expandafter{\the\everymath \color{myblue}}
\everydisplay\expandafter{\the\everydisplay \color{myblue}}

\renewcommand{\baselinestretch}{.8}
\pagestyle{empty}

\global\mdfdefinestyle{header}{%
linecolor=gray,linewidth=1pt,%
%leftmargin=1mm,rightmargin=2mm,skipbelow=2mm,skipabove=2mm,
}

\newcommand{\header}{
\begin{mdframed}[]
\footnotesize
\sffamily
\Large{\textbf{Julia}} \footnotesize Cheatsheet\\
by~Blair~Labatt~III,~page~\thepage~of~2
\end{mdframed}
}

\newenvironment{code}
    {
    \fboxsep0pt
    \colorbox{gray!30}
    }
    {
    }

\newcommand{\entry}[3]{%%%
    \begin{minipage}{{#1}}%%
        \code{#2}%%%
    \end{minipage}%%%
    \# #3
}%%%


%%%%%%%%%%%%%%%%%%%%%%%%%%%%%%%%%%%%%%%%%%%%%%
%% API command for columns of modules       %%
%%%%%%%%%%%%%%%%%%%%%%%%%%%%%%%%%%%%%%%%%%%%%%
\makeatletter 
\newcommand{\api}{%
  \begin{footnotesize} \color{blue}
  \@apii
}
\newcommand\@apii{\@ifnextchar\stopapi{\@apiend}{\@apiii}}

\newcommand\@apiii[2]{%
  \@apiiii{#1}{#2}\hfill
  \@apii % restart the recursion
}
\newcommand\@apiiii[2]{%
  \begin{minipage}[t]{{#1}}
  {#2}
  \end{minipage}}
\newcommand\@apiend[1]{% The argument is \stopapi
  \end{footnotesize}
}
\makeatother
%%%%%%%%%%%%%%%%%%%%%%%%%%%%%%%%%%%%%%%%%%%%%%%
%%%%%%%%%%%%%%%%%%%%%%%%%%%%%%%%%%%%%%%%%%%%%%%

 
\lstset{%
  basicstyle=\small\ttfamily,
  breaklines=false,
  backgroundcolor = \color{pastelgrey},
  language=[LaTeX]{TeX}
}

\makeatletter
\renewcommand{\section}{\@startsection{section}{1}{0mm}%
                                {1.6ex}%
                                {1.2ex}%x
                                {\color{myblue}\sffamily\Large\bfseries}}
\renewcommand{\subsection}{\@startsection{subsection}{1}{0mm}%
                                {.8ex}%
                                {.2ex}%x
                                {\sffamily\bfseries}}
\newcommand{\mycol}[1]{%%%
    \begin{minipage}{12mm}%%
        #1%%%
    \end{minipage}%%%
}%%%

\newcommand{\mycolX}[2]{%%%
    \begin{minipage}{{#1}}%%
        #2%%%
    \end{minipage}%%%
}%%%
\newcommand{\matr}[1]{\mathbf{#1}} 


\makeatother
\setlength{\parindent}{0pt}

\begin{document}
\small
\begin{multicols*}{4}
\header
\section{Invocation}

%%%%%%%%%%%%%%%%%%%%%%%%%%%%%%%%%%%%%%%%%%%
\subsection*{Environment Variables}
\textit{Edit \texttt{\textasciitilde .juliarc} for persistent $\Delta$s. Other env-variables for debugging, REPL color scheme, \& parallelization control \href{https://docs.julialang.org/en/v1/manual/environment-variables/index.html}{here}.}
\begin{itemize}
    \item JULIA\_EDITOR
    \item JULIA\_PROJECT
    \item JULIA\_LOAD\_PATH
    \item PLOTS\_DEFAULT\_BACKEND
\end{itemize}


%%%%%%%%%%%%%%%%%%%%%%%%%%%%%%%%%%%%%%%%%%%
\subsection*{REPL}
\mycolX{25mm}{\code{varinfo()}} \# inspect vars in namespace 



%%%%%%%%%%%%%%%%%%%%%%%%%%%%%%%%%%%%%%%%%%%
\subsection*{Scripts}
\mycolX{35mm}{\code{include(myfile.jl)}} \# run script \\
\mycolX{35mm}{\code{\#!/usr/bin/julia}} \# shell she-bang \\

\section{Basic Syntax}

\mycolX{35mm}{\code{x = 0}} \# simple assignment \\
\mycolX{35mm}{\code{global x += 1}} \# variable scoping \\


%%%%%%%%%%%%%%%%%%%%%%%%%%%%%%%%%%%%%%%%%%%%%%%%
\subsection*{Conditionals}
\mycolX{35mm}{\code{if 1==1}} \# if w/o paren \\
\mycolX{35mm}{\code{\phantom{xxx}...}} \# logic \\
\mycolX{35mm}{\code{end}} \# condit\textquotesingle ls all \say{end}\\
\mycolX{35mm}{\code{if <cnd>; <stmt>; end}} \# single line \\
\mycolX{45mm}{\code{z = <cnd> ? <exp1> : <exp2>}} \# ternary \\
\mycolX{35mm}{\code{for <cnd>; <stmt>*; end}} \# for syntax\\
\mycolX{35mm}{\code{for i=1:5,j=1:10}} \# multi-condition; ...\\




%%%%%%%%%%%%%%%%%%%%%%%%%%%%%%%%%%%%%%%%%%%%%%%%
\subsection*{Functions}
\textit{Functions do not exist as methods on \say{receiver} objects, but rather pass all objects / values as parameters. Polymorphism is accomplished at run-time through \say{multiple dispatch}, as determined by passed parameter types. Inquire into polymorphic overrides with \texttt{methods(myFn)} call.}\\
\mycolX{30mm}{\code{function f(x,y)}}\\
\mycolX{35mm}{\code{\phantom{xx}println(\textquotedbl Sum: \$x, \$y\textquotedbl)}} \# side-effects \\
\mycolX{35mm}{\code{\phantom{xx}if(x+y>100) return 5}} \# explicit return \\
\mycolX{35mm}{\code{\phantom{xx}x + y}} \# implicit return \\
\mycolX{30mm}{\code{end}}\\
\mycolX{30mm}{\code{f(x,y) = x + y}} \# $\approx$ to above\\
\mycolX{30mm}{\code{g = f;}} \# functions as variables\\
\mycolX{30mm}{\code{(x) -> x+a}} \# anonymous function \\
\mycolX{30mm}{\code{f(x,dim=2)}} \# named arguments \\
\mycolX{30mm}{\code{f(x,c...)}} \# spread operator \\
\mycolX{30mm}{\code{$\sum$(x,y) = x + y}} \# unicode in fn names\\[2mm]
\textit{By convention, exclamation indicates call by reference on 1\textsuperscript{st} argument.} \\
\mycolX{30mm}{\code{f!(rcvr,<arg>*);}} \# receiver free to mutate\\



%%%%%%%%%%%%%%%%%%%%%%%%%%%%%%%%%%%%%%%%%%%%%%%%
\subsection*{Operators}
\mycolX{30mm}{\code{+(1,2,3)}} \# operators are fns \\
\mycolX{30mm}{\code{$>>\hspace{3mm}<<$}} \# logical shifts\\
\mycolX{30mm}{\code{$>>>\hspace{3mm}<<<$}} \# arithm. shift \\
\mycolX{30mm}{\code{$\veebar$}} \# \textbackslash xor | \textbackslash veebar \\
\mycolX{30mm}{\code{==}} \# comparison \\
\mycolX{30mm}{\code{===}} \# object equality \\


%%%%%%%%%%%%%%%%%%%%%%%%%%%%%%%%%%%%%%%%%%%%%%%%
\subsection*{Higher Order Functions}
\mycolX{30mm}{\code{filter(z -> z>3, x)}} \# filter \\
\mycolX{30mm}{\code{map(z -> z\textasciicircum 2, [1,2,3])}} \# map \\
\mycolX{38mm}{\code{broadcast(myFn,myArray)}} \# $\approx$ map\\
\mycolX{30mm}{\code{f = x -> x\textasciicircum 2}} \# anonymous fn\\
\mycolX{30mm}{\code{myFn.(myArray)}} \# sugar for broadcast \\
\mycolX{30mm}{\code{reduce(+, xs, 5)}} \# $5 + \sum_i xs_i$ \\



%%%%%%%%%%%%%%%%%%%%%%%%%%%%%%%%%%%%%%%%%%%%%%%%
\subsection*{Comments}
\mycolX{30mm}{\code{\textbackslash \# comment text }} \# single line \\
\mycolX{30mm}{\code{\textbackslash \#= outer}} \# multi-line \\
\mycolX{30mm}{\code{\phantom{xxx} \textbackslash \#= inner ...}} \# nestable \\
\mycolX{30mm}{\code{=\textbackslash \# outer  =\textbackslash \#}} \# end of comments \\



%%%%%%%%%%%%%%%%%%%%%%%%%%%%%%%%%%%%%%%%%%%%%%%%
\subsection*{Exceptions}
\code{try ...; catch e; end; finally ...;} \\


%%%%%%%%%%%%%%%%%%%%%%%%%%%%%%%%%%%%%%%%%%%%%%%%
\subsection*{Macros}
\mycolX{35mm}{\code{@time mean(y)}} \# time mean fn \\
\mycolX{35mm}{\code{@code\_llvm f(1)}} \# look at asmb \\
\mycolX{40mm}{\code{@show cdf(Normal(0,1),.5)}} \# evaluate \\
\mycolX{30mm}{\code{@which copy([1,2,3])}} \# inspect mult dispatch \\
\mycolX{35mm}{\code{@benchmark fn()}} \# basic code profile \\
\mycolX{35mm}{\code{@profile fn()}} \# n simulation runs \\
\mycolX{35mm}{\code{@assert}} \# assert \\
\mycolX{35mm}{\code{@debug}} \# debug \\
\section{Collections}

%%%%%%%%%%%%%%%%%%%%%%%%%%%%%%%%%%%%%
\subsection*{Arrays}
\textit{$\exists$ numerous ways to create:}\\
\mycolX{30mm}{\code{A = [1 2 3 ...]}} \# concrete \\
\mycolX{30mm}{\code{A = [1:1000]}} \# comprehension sugar \\
\mycolX{30mm}{\code{A = [1 2; 3 4]}} \# matrix-style \\
\mycolX{30mm}{\code{A = j:k:n}} \# lazy init, step k \\
\mycolX{30mm}{\code{[f(i) for i in 1:10]}} \# comprehension \\
\mycolX{30mm}{\code{vcat(x,y,z)}} \# = [x;y;z] \\
\mycolX{30mm}{\code{hcat(x,y,z)}} \# = [x y a]  \\[2mm]
\textit{Incremental creation using:}\\
\begin{tabular}{l l l l}
    push!       & pop!      & append! & prepend! \\
    deleteat!   & pushfirst! & fill! & insert! \\
\end{tabular} \ \\

\textit{Array assignment is only a reference. Use \texttt{copy()} or \texttt{deepcopy()} to make a 1-layer, or full copy, respectively.}\\
\mycolX{30mm}{\code{B = A}} \# reference-only \\
\mycolX{30mm}{\code{B = copy(A)}} \# shallow copy \\[2mm]
\textit{Homogeneous arrays (same type) are fast; heterogeneous arrays (::Any) are possible. Can also create a Union of enumerated types:}\\
\mycolX{30mm}{\code{B = Union\{Int64,Float64\}[..]}} \\[2mm]
\textit{Array inquiry functions:}\\
\begin{tabular}{l l l}
    sort    &       isempty         & findall \\
    eltype & collect & size \\
\end{tabular} \

\textit{Operate over arrays with:}\\
{\scriptsize
\begin{tabular}{l l l l}
    sum & product & any & all \\
    minimum & maximum & findmin & findmax \\
    first & last & count & getindex \\
    filter & map & reduce & mapreduce \\
\end{tabular}} \

\textit{Transform arrays with:}\\
{\scriptsize
\begin{tabular}{l l l l}
    splice! & reverse! & sort! & zip \\
\end{tabular}} \ \\

%%%%%%%%%%%%%%%%%%%%%%%%%%%%%%%%%%
\subsection*{Strings}
\begin{itemize}
    \item \mycolX{15mm}{s1*s2} \# concatenate s1, s2
    \item \mycolX{15mm}{s\textasciicircum n} \# repeat s n times
    \item \mycolX{15mm}{\$\{var\}} \# interpolation
\end{itemize} \
\textit{Helper operations:}\\
\begin{tabular}{l l l l}
    join      & replace       & [l|r]pad      & [l|r]strip \\
    search      & rsearch      & in    &  \\
    index      & rindex      & beginswith    & endswith \\
    isalnum & isalpha & isascii & isblank \\
    isdigit & isgraph & islower & isprint \\
    isspace & ispunct & isupper & isxdigit \\
\end{tabular} \


%%%%%%%%%%%%%%%%%%%%%%%%%%%%%%%%%%%%%
\subsection*{Dictionaries}
\textit{Dictionaries are mutable and type-unstable if different types are injected. Basic syntax:}\\
\mycolX{35mm}{\code{myDict = Dict(}} \# declare ...\\
\mycolX{35mm}{\code{\phantom{xxx}\textquotesingle a\textquotesingle => 1}} \# assign ... \\
\mycolX{35mm}{\code{\phantom{xxx}\textquotesingle b\textquotesingle => 2 )}} \\
\mycolX{35mm}{\code{[d(i)=value for (i,value)]}} \# comprehension \\
\textit{Helper methods:} \\
\begin{tabular}{l l l l}
    get     &    getkey  &       values         & keys \\
    collect    &       haskey         & in \\
    length  &   delete!    & pop!   & merge \\
\end{tabular} \


%%%%%%%%%%%%%%%%%%%%%%%%%%%%%%%%%%%%%
\subsection*{Tuples}
\mycolX{30mm}{\code{a = (1,2,3)}} \# creating \\
\mycolX{30mm}{\code{a = tuple(1,2,3)}} \# ditto \\
\mycolX{30mm}{\code{a = ntuple(n,f)}} \# \underline{f}unction gen \\
\mycolX{30mm}{\code{a, b = (1,2)}} \# destructuring\\
\mycolX{30mm}{\code{tup = (a=1,b=2)}} \# \say{named} tuples\\
\mycolX{30mm}{\code{tup[1]; tup[2]}} \# indexing \\
\mycolX{30mm}{\code{first(tup); last(tup)}} \# ibid \\
\mycolX{30mm}{\code{tup.a}} \# dot-notation ($\equiv$ above) \\
\textit{Tuples are immutable. $\exists$ the following helpers:}\\
\begin{tabular}{l l l l}
    values      & keys      & pairs    & collect \\
\end{tabular} \



%%%%%%%%%%%%%%%%%%%%%%%%%%%%%%%%%%%%%%%%%%%%%%%%
\subsection*{Sets}
\mycolX{30mm}{\code{s = Set([1,2,3,...])}} \# creation \\
\mycolX{30mm}{\code{i = IntSet([1,2,3,...])}} \# sorted ints \\
\mycolX{30mm}{\code{intersect(s1,s2)}} \# s1 $\land$ s2 \\
\mycolX{30mm}{\code{union(s1,s2)}} \# s1 $\lor$ s2 \\
\mycolX{30mm}{\code{setdiff(s1,s2)}} \# s1 $\neg$ s2 \\
\mycolX{30mm}{\code{symdiff(s1,s2)}} \# s1 $\oplus$ s2 \\\mycolX{30mm}{\code{issubset(s1,s2)}} \# s1 $\subset$ s2 \\[2mm]
\textit{Additional helper methods:}\\
\begin{tabular}{l l l l}
    add! & complement! \\
\end{tabular} \
\section{Types}

%%%%%%%%%%%%%%%%%%%%%%%%%%%%%%%%%%
\subsection*{Missing}
\begin{itemize}
    \item missing::Missing
    \item nothing::Nothing
    \item NaN::Float64
    \item Inf
\end{itemize} \


%%%%%%%%%%%%%%%%%%%%%%%%%%%%%%%%%%
\subsection*{Numeric Types}
\begin{itemize}
    \item Int64 ...... 42
    \item Float64 ...... 0.2, 1e10, 4.
    \item Char ....... 'a','b'
    \item Bool ....... true, false
    \item Complex ...... 5-2im, complex(5,2)
\end{itemize} \

\textit{Basic math functions:}\\
\begin{tabular}{l l l l}
    abs & cmp & round & divrem   \\
    real & imag \\
\end{tabular} \

%%%%%%%%%%%%%%%%%%%%%%%%%%%%%%%%%%%%%
\subsection*{Structs}
\mycolX{40mm}{\code{mutable struct MyS\{T<:Number\}}}  \\
\mycolX{40mm}{\code{\phantom{xxx}property1}} \# untyped \\
\mycolX{40mm}{\code{\phantom{xxx}property2::String}} \# concrete type \\
\mycolX{40mm}{\code{\phantom{xxx}property3::T}} \# type constraint \\
\mycolX{40mm}{\code{end}} \#  \\
\mycolX{40mm}{\code{mutable struct MyMutS ...}} \# mutable \\
\mycolX{40mm}{\code{s = MyS(\textquotedbl a\textquotedbl,\textquotedbl b\textquotedbl,5)}} \# initialization \\
\mycolX{40mm}{\code{a = s.property3}} \# referencing \\
\textit{To define abstract types:}\\
\mycolX{40mm}{\code{abstract type MyGenType end}}  \\
\mycolX{40mm}{\code{abstract type MyConcType <:MyGenType end}}  \\


%%%%%%%%%%%%%%%%%%%%%%%%%%%%%%%%%%%%%
\subsection*{\href{https://docs.julialang.org/en/v1/manual/constructors/}{Constructors}}
\textit{Technically ``methods'', $\exists$ two types: `inner' and `outer'. An eponymous `default' inner is provided:}\\
\entry{30mm}{struct MyStruct }{``composite'' type}\\
\entry{30mm}{\phantom{xxx}x::Int64 end}{a single field}\\
\entry{30mm}{s = MyStruct(5)}{default constructor}\\[2mm]
\textit{Inners are used for enforcing invariants:}\\
\code{struct MyControlledStruct}\\
\code{\phantom{xxx}MyControlledStruct(x::Int64) = \textbackslash}\\
\code{\phantom{xxxxxx}x > 5 ? new(x) : error("must be >5")}\\
\code{end}\\[2mm]
\textit{Outers can be appended by other modules, and are used for convenience (eg, argument elision):}\\
\entry{38mm}{MyStruct() = MyStruct(3)}{use mult. dispch}\\[2mm]
\textit{Parametric types JIT create concrete types with an inner constructor. Invoke implicitly or explicitly:}\\
\entry{30mm}{p = Point(1.0,2.5)}{implicit: T = Float64}\\
\entry{30mm}{p = Point\{Int64\}(1,2)}{explicit T = Int64}\\



%%%%%%%%%%%%%%%%%%%%%%%%%%%%%%%%%%
\subsection*{Regexs}
\mycolX{35mm}{\code{rm = match(r"regex",s,i)}} \# execute \\
\mycolX{35mm}{\code{\phantom{xxx}rm.match}} \# substring matched \\
\mycolX{35mm}{\code{\phantom{xxx}rm.captures}} \# tuple of matches \\
\mycolX{35mm}{\code{\phantom{xxx}rm.offsets}} \# vector of matches \\




%%%%%%%%%%%%%%%%%%%%%%%%%%%%%%%%%%
\subsection*{Conversion}
\mycolX{35mm}{\code{parse(Float64, "3.14")}} \# float from string \\
\mycolX{35mm}{\code{float64("3.14")}} \# ditto \\
\mycolX{35mm}{\code{string(3.14)}} \# inverse \\
\mycolX{35mm}{\code{int8("123")}} \# int from string \\
\mycolX{35mm}{\code{hex(x); oct(x); dec(x)}} \# various casts \\


%%%%%%%%%%%%%%%%%%%%%%%%%%%%%%%%%%
\subsection*{Type System \& Generics}
\mycolX{35mm}{\code{q::Number}} \# type annotation \\
\mycolX{35mm}{\code{f\{T<:Number\}(x::T,y::T)}} \# parametric types \\
\mycolX{35mm}{\code{f\{A<:B\}(x::A)}} \# subtype constraints \\




%%%%%%%%%%%%%%%%%%%%%%%%%%%%%%%%%%
\subsection*{Type System Helper Methods}
\begin{itemize}
    \item subtypes(type)
    \item supertype(type)
    \item fieldnames(type)
    \item isa(field,type)
    \item typeof(obj)
    \item isequal(x,y)
\end{itemize}

\section{File I/O}

\textit{open() a file, returning a handler (passing a \say{modality} $\in$ \underline{r}ead, \underline{w}rite, \underline{a}ppend); then close.}\\
\mycolX{35mm}{\code{h = open(\textquotedbl f.txt\textquotedbl,\textquotedbl r\textquotedbl )}} \# create \underline{h}andle \\
\mycolX{35mm}{\code{cont = read(f,String)}} \# read \\
\mycolX{35mm}{\code{close(h)}} \# close \\


\textit{It is, however, idiomatic to use the \say{do} construct:}\\
\mycolX{35mm}{\code{open(\textquotedbl f.txt\textquotedbl ,\textquotedbl r\textquotedbl) do h}} \# create \\
\mycolX{35mm}{\code{\phantom{xxx} cont = read(h,String)}} \# read \\
\mycolX{35mm}{\code{end}} \# implied close \\
\


\textit{Reading line by line:}\\
\mycolX{35mm}{\code{open(...) do h}} \#  \\
\mycolX{35mm}{\code{\phantom{xx}for ln in eachline(h)}} \#  \\
\mycolX{35mm}{\code{\phantom{xxxx}println(ln)}} \#  \\
\mycolX{35mm}{\code{\phantom{xx}end}} \#  \\
\mycolX{35mm}{\code{end}} \#  \\

\textit{Writing:}\\
\mycolX{35mm}{\code{open(\textquotedbl f.txt\textquotedbl,\textquotedbl w\textquotedbl) do h}} \#  \\
\mycolX{35mm}{\code{\phantom{xxx}write(h,\textquotedbl text\textbackslash n\textquotedbl)}} \#  \\
\mycolX{35mm}{\code{end}} \#  \\


\textit{Helper methods:}\\
{\scriptsize 
\begin{tabular}{l l l l}
    position & seek & seekstart & seekend \\
    skip & isopen & oef & isreadonly \\
    ltoh & ltol & [de]serialize &  download \\
    readbytes & readcsv & readall & readlines \\
\end{tabular}} \ \
%\section{\href{https://docs.juliaplots.org/latest/input\_data/}{Visualization}}

\textit{Frontends: \href{http://docs.juliaplots.org/latest/}{Plots} (\href{http://gadflyjl.org/stable/}{Gadfly},  \href{https://github.com/JuliaGraphics/Winston.jl}{Winston} are obsolete). \say{Backend} packages for Plots: \href{https://github.com/plotly/Plotly.jl}{PlotlyJS}, \href{https://github.com/JuliaPy/PyPlot.jl}{PyPlot}, \href{https://github.com/jheinen/GR.jl}{GR}, \href{https://kristofferc.github.io/PGFPlotsX.jl/stable/}{PGFPlotsX}. (Eg, \texttt{\code{gr()}}). See \href{https://docs.juliaplots.org/latest/generated/supported/}{here} for attributes supported per backend.}


%%%%%%%%%%%%%%%%%%%%%%%%%%%%%%%%%%%%%%%%%%
\subsection*{\href{http://docs.juliaplots.org/latest/}{Plots}}
\mycolX{30mm}{\code{p = plot(x,y)}} \# visual side-effect \\
\mycolX{30mm}{\code{plot!(x,y)}} \# add to current plot \\
\mycolX{30mm}{\code{plot(p,x,y)}} \# $\equiv$ to above (adds to p) \\
\mycolX{35mm}{\code{z = rand(10,2); plot(x,z)}} \# mult series \\
\mycolX{30mm}{\code{plotly()}} \# set plotly backend \\
\mycolX{30mm}{\code{gr()}} \# set gr backend \\
\mycolX{30mm}{\code{display(plot(x, y))}} \# required in scripts\\
\mycolX{30mm}{\code{plot(x,x->sin(x))}} \# plot anon fn \\
\mycolX{40mm}{\code{tvec = range(0, 6.28, length = 100)}}  \\
\mycolX{30mm}{\code{plot(sin,cos,tvec)}} \# parametric plot \\


%%%%%%%%%%%%%%%%%%%%%%%%%%%%%%%%%%%%%%%%%%
\subsection*{Stylizing}
\textit{Choose a a color scheme from the many \href{https://docs.juliaplots.org/latest/generated/plotthemes/}{here}, a theme from \href{https://docs.juliaplots.org/latest/generated/plotthemes/}{here}, and then fine-tune by hand like:} \\
\mycolX{48mm}{\code{plot!(p,title="...")}} \# title \\
\mycolX{48mm}{\code{plot!(p,label = [\textquotedbl Line 1\textquotedbl \hspace{1mm} \textquotedbl Line 2\textquotedbl])}} \# attrs \\
\mycolX{48mm}{\code{xlabel!("My x")}} \# alternat. \\
\mycolX{25mm}{\code{clibrary(<clib>)}} \# import color library $\in$: \\ 
\textit{:Plots, :cmocean, :misc, :colorcet, :colorbrewer} \\
\mycolX{30mm}{\code{plotattr()}} \# query params \\

\textit{Plot-level attributes (more \href{https://docs.juliaplots.org/latest/generated/attributes\_plot/}{here}):}\\
{\scriptsize
\begin{tabular}{l l l}
    bg (color)      & size          & dpi \\
    fontfamily      & title         & legend \\
    framestyle      & aspect\_ratio & camera \\
    palette         \\
\end{tabular}
}\\
\textit{Grid attributes (more \href{https://docs.juliaplots.org/latest/generated/attributes\_plot/}{here}):}\\
{\scriptsize
\begin{tabular}{l l l}
    grid            & gridlinewidth         & link \\
    {[x|y|z]}lims     & [x|y|z]ticks          & [x|y|z]scale\\
    {[x|y]}guide      & [x|y]label \\
\end{tabular}
}

\textit{Series-level attributes (more \href{https://docs.juliaplots.org/latest/generated/attributes\_series/}{here}):}\\
{\scriptsize
\begin{tabular}{l l l}
    Points                  & Lines          & Surfaces \\ \hline
    markercolor             & linecolor     & fillrange \\
    markeralpha             & linealpha     & fillcolor \\
    markersize              & linestyle     & fillalpha \\
    markershape             & linewidth     & \\
    markerstroke-           & \\
    \phantom{xxxx}-color    & \\
    \phantom{xxxx}-alpha    & \\
    \phantom{xxxx}-width    & \\
\end{tabular}
}

%%%%%%%%%%%%%%%%%%%%%%%%%%%%%%%%%%%%%%%%%%
\subsection*{Plot Types}
\mycolX{45mm}{\code{plot!(p,seriestype = :scatter)}} \# series \\
\textit{Where seriestype $\in$:}\\
{\scriptsize
\begin{tabular}{l l l}
    line        & path          & steppre \\
    steppost    & sticks        & scatter \\
    heatmap     & hexbin        & barbins \\
    barhist     & stephist      & bins2d \\
    histogram2d & histogram3d   & density \\
    bar         & hline         & vline \\
    contour     & pie           & shape \\
    image       & path3d        & scatter3d \\
    surface     & wireframe     & contour3d \\
    volume  \\
\end{tabular}
}


%%%%%%%%%%%%%%%%%%%%%%%%%%%%%%%%%%%%%%%%%%
\subsection*{Screen Layout}
\mycolX{43mm}{\code{plot(x, y, layout = (4, 1))}} \# 4x1 vertically \\
\mycolX{43mm}{\code{plot(p1,p2,p3,p4,layout=(2,2))}} \# saved p\textquotesingle s\\
\mycolX{43mm}{\code{l = @layout [a\{0.6h\} b\{0.6w\} c]}} \# advanced \\
\mycolX{43mm}{\code{plot(x, y, layout = l)}} \# use above \\
\mycolX{35mm}{\code{BB = (x1,x2,y1,y2)}} \# set boundingbox \\
\mycolX{35mm}{\code{plot(x,y,inset=(1,BB))}} \# insetting \\


%%%%%%%%%%%%%%%%%%%%%%%%%%%%%%%%%%%%%%%%%%
\subsection*{Exporting \& Importing}
\textit{Save to .eps, .html, .pdf, .png, .ps, .svg, .tex, .text:}\\
\mycolX{35mm}{\code{savefig(\textquotedbl myplot.png\textquotedbl)}} \# from screen \\
\mycolX{35mm}{\code{savefig(p, \textquotedbl myplot.pdf\textquotedbl)}} \# from var p \\
\mycolX{35mm}{\code{png(fn)}} \# shorthand save as \\
\mycolX{35mm}{\code{img = load(\textquotedbl a.png\textquotedbl)}} \# load image \\
\mycolX{35mm}{\code{plot(x,y,img)}} \# plot an image \\


%%%%%%%%%%%%%%%%%%%%%%%%%%%%%%%%%%%%%%%%%%
\subsection*{Animations}
\textit{See \href{https://docs.juliaplots.org/latest/\#simple-is-beautiful-1}{here} for more examples.}\\
\mycolX{35mm}{\code{p = plot([sin, cos], zeros(0), leg = false)}} \\
\mycolX{35mm}{\code{anim = Animation()}} \\
\mycolX{35mm}{\code{for x = range(0, stop = 10$\pi$, length = 100)}}  \\
\phantom{xxx}\mycolX{35mm}{\code{push!(p, x, Float64[sin(x), cos(x)])}}   \\
\phantom{xxx}\mycolX{35mm}{\code{frame(anim)}} \\
\mycolX{35mm}{\code{end}} \\




%%%%%%%%%%%%%%%%%%%%%%%%%%%%%%%%%%%%%%%%%%
\subsection*{Extensions}
\textit{Use or create \href{http://docs.juliaplots.org/latest/recipes/\#recipes-1}{recipes} for often-generated plot-types. Eg, \href{https://github.com/JuliaPlots/StatsPlots.jl}{StatsPlot} allows visualization of data frames, distributions, boxplots, etc. Also see \href{https://github.com/JuliaPlots/GraphRecipes.jl}{GraphRecipes} for help plotting graphs. Alternatively, browse the \say{\href{http://docs.juliaplots.org/latest/ecosystem/\#ecosystem-1}{ecosystem}}.}


\mycolX{35mm}{\code{@df iris scatter( }} \# using Dataframes\\
\mycolX{35mm}{\code{    :SepalLength, :SepalWidth)}} \\ 
\mycolX{35mm}{\code{plot(Normal(3, 5))}} \# using Distributions  \\

%\mycolX{35mm}{\code{  }} \#  \\
 %%% OLD -- moved to "plotting" cheatsheet!
\section{\href{https://docs.julialang.org/en/v1/stdlib/LinearAlgebra/}{Linear Algebra}}

%%%%%%%%%%%%%%%%%%%%%%%%%%%%%%%%%%%%%%%%%%%
\subsection*{Building Matrices}
\mycolX{35mm}{\code{[1. 2. 3.; 4. 5. 6.]}} \# 2x3 matrix $\in \mathbb{Q}$\\
\mycolX{35mm}{\code{A = [1 2 3]'}} \# transpose \\
\mycolX{35mm}{\code{transpose(A)}} \# ibid \\ 
\mycolX{35mm}{\code{reshape(A,dims)}} \# transform \\ \mycolX{35mm}{\code{A = ones(2, 2)}} \# 2x2 matrix \\
\mycolX{35mm}{\code{A = zeros(2,2)}} \# 2x2 $\matr{0}-matrix$\\
\mycolX{35mm}{\code{A = Diagonal(A)}} \# diag. of $\matr{A}$\\
\mycolX{30mm}{\code{A = I}} \# Identity matrix\\
\mycolX{35mm}{\code{A = Matrix\{Int\}(I, 3, 3)}} \# ibid, 3x3 of ints\\
\mycolX{35mm}{\code{A = reshape(1:10,5,2)}} \# shape from linear \\
\mycolX{35mm}{\code{C = similar(A)}} \# same dims \\

%%%%%%%%%%%%%%%%%%%%%%%%%%%%%%%%%%%%%%%%%%%
\subsection*{Indexing into Matrices}
\mycolX{35mm}{\code{A[2,2]}} \# access element \\
\mycolX{35mm}{\code{A[1:4,:]}} \# access rows \\
\mycolX{35mm}{\code{A[:,1:4]}} \# access cols \\
\\\mycolX{35mm}{\code{A[[1,2,4],:]}} \# deselect row \\
\mycolX{35mm}{\code{diag(A)}} \# retrieve diagonal \\
\mycolX{35mm}{\code{size(A)}} \# get dimensions \\



%%%%%%%%%%%%%%%%%%%%%%%%%%%%%%%%%%%%%%%%%%%
\subsection*{Matrix Math}
\mycolX{35mm}{\code{eigvals(A)}} \# eigenvalues \\
\mycolX{35mm}{\code{eigvect(A)}} \# eigenvectors \\
\mycolX{35mm}{\code{inv(A)}} \# inverse \\
\mycolX{35mm}{\code{det(A)}} \# determinant \\
\mycolX{35mm}{\code{A .* B}} \# element mult \\
\mycolX{35mm}{\code{A * B}} \# matrix mult \\
\mycolX{35mm}{\code{dot(v1,v2)}} \# vector dot prod \\
\mycolX{35mm}{\code{A\textbackslash b}} \# solve Ax = b \\
\mycolX{35mm}{\code{rref(A)}} \# \href{https://juliapackages.com/packages/rowechelon}{\ul{r}ed. \ul{r}ow-\ul{e}chelon} \\
\mycolX{35mm}{\code{nullspace(A)}} \# nullspace \\

\section{\href{https://docs.julialang.org/en/v1/stdlib/Statistics/}{Statistics} \& \href{https://juliastats.org/Distributions.jl/stable/}{Probability}}
\textit{Import \say{Statistics}, \say{BaseStats}, \say{Distributions}.}


%%%%%%%%%%%%%%%%%%%%%%%%%%%%%%%%%%%%%%%%%%%%%
\subsection*{Random Numbers}
\mycolX{35mm}{\code{A = rand(2)}} \# 2 rand floats \\
\mycolX{35mm}{\code{A = rand(2,2)}} \# 2x2 matrix $\in \mathbb{Q}$\\
\mycolX{35mm}{\code{rand(Uniform(a,b),2,3)}} \# from distribution \\
%\mycolX{35mm}{\code{rand(a:.1:b)}} \# .1 precision \\


%%%%%%%%%%%%%%%%%%%%%%%%%%%%%%%%%%%%%%%%%%%%%
\subsection*{Array / Matrix Reducers}
\mycolX{30mm}{\code{sum(A,dims=2)}} \# sum rows (dim=2) \\
\mycolX{30mm}{\code{max(A,dims=1)}} \# max by cols (dim=1) \\
\mycolX{30mm}{\code{min(A,dims=2)}} \# min by rows \\
\mycolX{40mm}{\code{cumsum(A,dims=2)}} \# cumulative $\sum$ \\
\mycolX{40mm}{\code{accumulate(max,A,dims=1)}} \# apply max fn \\


%%%%%%%%%%%%%%%%%%%%%%%%%%%%%%%%%%%%%%%%%%%%%
\subsection*{Probability Distributions}

\entry{32mm}{D = Normal(0,1)}{a std normal}\\
\entry{32mm}{quantile(D,[.9,.95])}{$\alpha = [.1,.05]$ quantiles}\\
\entry{32mm}{cdf(D,x)}{$F_{\mu,\sigma^2}(x)$\, (here $\Phi(x)$)}\\
\entry{32mm}{Binomial(.5)}{other distribt'ns}\\
\entry{32mm}{fit(D,x)}{generic fit}\\
\entry{32mm}{fit\_mle(D, x)}{$\hat{\theta}^{\mathrm{MLE}}$ estimation}\\
%\entry{35mm}{}{}\\
\textit{$\exists$ many dists, eg: \href{https://juliastats.org/Distributions.jl/stable/univariate/}{uni-} \& \href{https://juliastats.org/Distributions.jl/stable/multivariate/}{multivariate}, \& \href{https://juliastats.org/Distributions.jl/stable/matrix/}{matrix}.}\\

\begin{comment}
\api
{1.9cm}{
Arcsine     \\
Beta[Prime] \\
Biweight    \\
Cauchy      \\
Chi[sq]     \\
Cosine      \\
Epanechnikov\\
Erlang      \\
Exponential \\
FDist       \\
Frechet     \\
Gamma       \\
Gen'lzdPareto\\
Gumbel      \\
VonMises    \\
}
{1.9cm}{
Inv.Gamma   \\
Inv.Gaussian\\
Kolmogorov  \\
KSDist      \\
KSOneSided  \\
Laplace     \\
Levy        \\
Locat'nScale\\
Logistic    \\
LogitNormal \\
LogNormal   \\
Noncent.Beta\\
Noncent.Chisq\\
Noncent.F   \\
Weibull     \\
}
{2.2cm}{
Normal      \\
Norm.Gaussian\\
Norm.Inv.Gauss.\\
Pareto      \\
Gen'lzdExt'mVal.\\
PGenl'zdGauss.\\
Rayleigh    \\
Semicircle  \\
StudentizedRange\\
SymTriang'lrR'ng\\
TDist       \\
TriangularDist\\
TriWeight   \\
Uniform     \\
}
\stopapi


\textit{Discrete:}\\
\api
{1.8cm}{
Bernoulli   \\
BetaBinomial\\
Binomial    \\
Categorical \\
}
{1.9cm}{
Dirac       \\
Disc.Uniform\\
Geometric   \\
Skellam     \\
}
{2.3cm}{
HyperGeometric\\
Disc.NonParamt'c\\
NegativeBinom'l\\
Poisson     \\
Poiss.Binomial\\

}
\stopapi



\textit{[Matrix] Multivariate:}\\
\api
{2cm}{
Multinomial \\
Abst'ctMvNormal\\
MvNormal    \\
MvNorm.Canon\\
}
{2cm}{
MvLogNormal \\
Dirichlet   \\
Product     \\
MatrixNormal\\
{[Inv.]}Wishart\\
}
{2cm}{
MatrixReshaped\\
MatrixTDist \\
MatrixBeta  \\
MatrixFDist \\
LKJ         \\
}
\stopapi
\end{comment}

%%%%%%%%%%%%%%%%%%%%%%%%%%%%%%%%%%%%%%%%%%%%%
\subsection*{Statistics}
\entry{30mm}{var(Normal())}{$\mathrm{var}\left(\mathcal{N}(0,1)\right)$}\\
\entry{30mm}{var([1,2,3])}{$\widetilde{S}(\vec{x})$ (here $=1$)}\\
\entry{30mm}{mean([1,2,3])}{mean, here $=2$}\\
\entry{32mm}{mod = @formula(y \textasciitilde x)}{modelling (see \href{https://www.machinelearningplus.com/linear-regression-in-julia/}{here})}\\
\entry{30mm}{lm(mod,data)}{regression}\\
%\entry{35mm}{}{}\\
\api
{1.4cm}{
max     \\
min     \\
extreme \\
cor     \\
}
{1.3cm}{
mean    \\
var     \\
std     \\
mode    \\
cf      \\
}
{1.9cm}{
skewn'ss\\
kurtosis\\
mgf     \\
pdfsq.n'rm\\
}
{1.4cm}{
cov     \\
invcov  \\
location\\
scale   \\
}
\stopapi

\section{\href{https://docs.sciml.ai/stable/}{Differential Equations}}
\textit{Import and use \say{DifferentialEquations}. Then:}
\begin{itemize}
    \item Define the problem (system, tspan, ICs)
    \item Solve the problem (parameterize solver)
    \item Analyze the solution (plot or inspect data)
\end{itemize}

\mycolX{35mm}{\code{f(t,u) = 1.01*u}} \# the system \\
\mycolX{35mm}{\code{u0=1/2}} \# initial condition \\
\mycolX{35mm}{\code{tspan = (0.0,1.0)}} \# timespan \\
\mycolX{45mm}{\code{prob = ODEProblem(f,u0,tspan)}} \# wrap \\
\mycolX{35mm}{\code{sol = solve(prob)}} \# solve \\
\mycolX{39mm}{\code{sol = solve(prob,reltol=1e-6)}} \# parameterize\\
\mycolX{35mm}{\code{sol = solve(prob,Tsit5())}} \# different solver \\
\mycolX{35mm}{\code{sol[5]}} \# 5th step value\\
\mycolX{35mm}{\code{sol.t[3]}} \# 3rd timestep val\\
\mycolX{35mm}{\code{sol(.45)}} \# interpolated val\\
\mycolX{35mm}{\code{plot(sol)}} \# plot \\[2mm]


%%%%%%%%%%%%%%%%%%%%%%%%%%%%%%%%%%%%%%%%%%%%%%%%
\subsection*{Systems of Equations}
\mycolX{35mm}{\code{function lorenz(t,u,du)}} \# lorentz eqn\\
\mycolX{35mm}{\code{ du[1] = 10.0(u[2]-u[1])}} \\
\mycolX{35mm}{\code{ du[2] = u[1]*(28.0-u[3]) - u[2]}} \\
\mycolX{35mm}{\code{ du[3] = u[1]*u[2] - (8/3)*u[3]}} \\
\mycolX{20mm}{\code{end}} \# now pass to ODEProblem()\\


\section{\href{https://github.com/JuliaPy/SymPy.jl}{Computer Algebra}}
%\textit{with following:}\\
\entry{30mm}{ENV[\textquotedbl PYTHON\textquotedbl]=\textquotedbl\textquotedbl}{use private Python}\\
{\footnotesize \code{Pkg.add(``PyCall''); Pkg.build(``PyCall''); using PyCall}}\\
\textit{then reload Julia. See \href{https://nbviewer.jupyter.org/github/sylvaticus/juliatutorial/blob/master/assets/Symbolic\%20computation.ipynb}{here} for a notebook, \href{https://juliahub.com/docs/SymPy/KzewI/1.0.28/Tutorial/intro/}{here} for a julia tutorial, or \href{https://docs.sympy.org/latest/tutorial/calculus.html}{here} for Python.}
\code{Pkg.add(``SymPy''); \quad using SymPy}\\

\entry{35mm}{x,y,z = symbols("x y z")}{define symbols}\\
\entry{35mm}{\@ vars x,y,z}{equiv to above}\\
\entry{35mm}{expr = x+2*y}{an expression}\\
\entry{35mm}{expand(x*(x+2*y))}{expanding}\\
\entry{35mm}{factor(x\textasciicircum 2+2*x*y)}{factoring}\\
\entry{35mm}{simplify(f-g)}{reduce}\\
\entry{47mm}{cancel((x\textasciicircum 2 + 2*x + 1)/(x\textasciicircum 2 + x))}{std form}\\
\entry{30mm}{apart(expr)}{partial frac. decomp.}\\
\entry{35mm}{f=(x+1)\textasciicircum 2;g = x\textasciicircum 2-2x+1}{functions}\\
\entry{35mm}{f-g}{higher order}\\
\entry{35mm}{expr(x=>1,y=>2)}{parameterize}\\
\entry{35mm}{expr.subs([(x,2), (y,2)])}{ibid}\\

%%%%%%%%%%%%%%%%%%%%%%%%%%%%%%%%%%%%%%%%%%%%%%%%
\subsection*{Solving}
\entry{35mm}{solve(expr,y)}{solve: expr=0}\\
\entry{35mm}{eqn = Eq(expr2,2)}{equation object}\\
\entry{35mm}{solve(eqn)}{solve arb. eqn}\\

%%%%%%%%%%%%%%%%%%%%%%%%%%%%%%%%%%%%%%%%%%%%%%%%
\subsection*{Calculus}
\entry{35mm}{diff(cos(x), x)}{differentiate wrt x}\\
\entry{35mm}{diff(x\textasciicircum 4, x, x, x)
}{$d^3/dx^3 (x^4)$}\\
\entry{35mm}{integrate(cos(x), x)}{indefinite}\\
\entry{38mm}{integrate(exp(-x), (x, 0, oo))}{definite: $\int_0^{\infty} e^x dx$}\\
\entry{50mm}{sympy.Integral(cos(x)\textasciicircum 2, (x, 0, PI))}{$\neg$ eval\textquotesingle d}\\
\entry{35mm}{limit(sin(x)/x, x, 0)}{limit}\\
\entry{35mm}{exp(sin(x)).series(x, 0, 4)}{series expand}\\

%%%%%%%%%%%%%%%%%%%%%%%%%%%%%%%%%%%%%%%%%%%%%%%%
\subsection*{Linear Algebra}
\entry{35mm}{M = Sym[1 2 3; 3 2 1]}{matrix}\\
\entry{35mm}{N = Sym[0, 1, 1]}{vector}\\
\entry{35mm}{M*N}{symbolic eval}\\
\entry{35mm}{det([1 x; y 4])}{LA lib. works}\\

%%%%%%%%%%%%%%%%%%%%%%%%%%%%%%%%%%%%%%%%%%%%%%%%
\subsection*{Miscellaneous}
\entry{35mm}{Or(x,y)}{logic}\\
\entry{50mm}{sympy.expand\_trig(sin(2x)+cos(2x))}{trig}\\
\entry{45mm}{trigsimp(sin(x)\textasciicircum 2 + cos(x)\textasciicircum 2)}{simplify}\\
\entry{35mm}{powsimp(x\textasciicircum a*x\textasciicircum b)}{power simplify}\\


%\section{Machine Learning}

%\section{Macros}



\section{Packages \& Modules}

\subsection*{Installing, Managing}
\textit{Can alternatively use \say{package} mode, hitting \say{]} at REPL, then \say{?} for commands, which include: add, update, status, rm, etc.}\\
\mycolX{40mm}{\code{Using Pkg}} \#  \\
\mycolX{40mm}{\code{Pkg.add("name")}} \#  \\
\mycolX{40mm}{\code{Pkg.clone("https://github...")}} \#  \\
\mycolX{40mm}{\code{Pkg.checkout(url)}} \#  \\[1mm]
\textit{After adding a package, you must explicitly include it with either }\texttt{import}\textit{ or }\texttt{using}:\\
%keeps namespace clean, necessitating full path resolution; }\texttt{using}\textit{ clutters namespace.}\\
\mycolX{30mm}{\code{import Plots }} \#  include (clean NS) \\
\mycolX{30mm}{\code{using Plots }} \#  ibid (cluttered NS) \\
\mycolX{30mm}{\code{require(file) }} \# load once \\
\mycolX{30mm}{\code{reload(file) }} \# reload \\
\mycolX{30mm}{\code{include(file) }} \# set dir, load \\



\subsection*{Common Packages}

\begin{itemize}
    \item GLM
    \item \mycolX{20mm}{\href{http://juliadata.github.io/DataFrames.jl/stable/}{DataFrames}} Tabular data manipulation 
    \item NullableArrays
    \item MixedModels
    \item Optim
    \item \mycolX{20mm}{\href{https://juliastats.github.io/Distributions.jl/stable/}{Distributions}} 
    \item \mycolX{20mm}{JuMP} Machine Learning
    \item \mycolX{20mm}{\href{https://juliahub.com/docs/SymPy/KzewI/1.0.25}{SymPy}} Symbolic computation
    \item Convex
    \item Losses
    \item Transformations
    \item jplyr
    \item Query
    \item SciKitLearn
    \item PyCall
    \item RCall
    \item \mycolX{20mm}{\href{https://juliahub.com/docs/Roots/o0Xsi/1.0.5/roots/}{Roots}} [Transcendental] Eqn Solver
    \item CSV
    \item \mycolX{20mm}{ODBC} Database connectivity
    \item Mamba
    \item HTTP
    \item \mycolX{20mm}{Weave} Document generation
    \item \mycolX{20mm}{OdsIO} Open Documents
    \item Images
\end{itemize}
\section{\href{https://plutojl.org/}{Pluto}}


%%%%%%%%%%%%%%%%%%%%%%%%%%%%%%%%%%%%%%%%%%%%%%%%
\subsection*{Basic Use}


%%%%%%%%%%%%%%%%%%%%%%%%%%%%%%%%%%%%%%%%%%%%%%%%
\subsection*{Keyboard Shortcuts}
\textit{Try customizing with \href{https://github.com/lucio-cornejo/custom-pluto}{this}.}

\begin{itemize}
	\item \code{Ctl + Alt + b} ?
	\item \code{Ctl + Alt + l} 
	\item \code{Ctl + Alt + v} toggle cell visibility 
	\item \code{Ctl + Alt + s} split cell
	\item \code{Ctl + Alt + enter} new cell above
\end{itemize}


%%%%%%%%%%%%%%%%%%%%%%%%%%%%%%%%%%%%%%%%%%%%%%%%
\subsection*{UI and HTML}





\vfil\,


\end{multicols*}
\end{document}
