\section{\href{https://docs.juliaplots.org/latest/input\_data/}{Visualization}}

\textit{Frontends: \href{http://docs.juliaplots.org/latest/}{Plots} (\href{http://gadflyjl.org/stable/}{Gadfly},  \href{https://github.com/JuliaGraphics/Winston.jl}{Winston} are obsolete). \say{Backend} packages for Plots: \href{https://github.com/plotly/Plotly.jl}{PlotlyJS}, \href{https://github.com/JuliaPy/PyPlot.jl}{PyPlot}, \href{https://github.com/jheinen/GR.jl}{GR}, \href{https://kristofferc.github.io/PGFPlotsX.jl/stable/}{PGFPlotsX}. (Eg, \texttt{\code{gr()}}). See \href{https://docs.juliaplots.org/latest/generated/supported/}{here} for attributes supported per backend.}


%%%%%%%%%%%%%%%%%%%%%%%%%%%%%%%%%%%%%%%%%%
\subsection*{\href{http://docs.juliaplots.org/latest/}{Plots}}
\mycolX{30mm}{\code{p = plot(x,y)}} \# visual side-effect \\
\mycolX{30mm}{\code{plot!(x,y)}} \# add to current plot \\
\mycolX{30mm}{\code{plot(p,x,y)}} \# $\equiv$ to above (adds to p) \\
\mycolX{35mm}{\code{z = rand(10,2); plot(x,z)}} \# mult series \\
\mycolX{30mm}{\code{plotly()}} \# set plotly backend \\
\mycolX{30mm}{\code{gr()}} \# set gr backend \\
\mycolX{30mm}{\code{display(plot(x, y))}} \# required in scripts\\
\mycolX{30mm}{\code{plot(x,x->sin(x))}} \# plot anon fn \\
\mycolX{40mm}{\code{tvec = range(0, 6.28, length = 100)}}  \\
\mycolX{30mm}{\code{plot(sin,cos,tvec)}} \# parametric plot \\


%%%%%%%%%%%%%%%%%%%%%%%%%%%%%%%%%%%%%%%%%%
\subsection*{Stylizing}
\textit{Choose a a color scheme from the many \href{https://docs.juliaplots.org/latest/generated/plotthemes/}{here}, a theme from \href{https://docs.juliaplots.org/latest/generated/plotthemes/}{here}, and then fine-tune by hand like:} \\
\mycolX{48mm}{\code{plot!(p,title="...")}} \# title \\
\mycolX{48mm}{\code{plot!(p,label = [\textquotedbl Line 1\textquotedbl \hspace{1mm} \textquotedbl Line 2\textquotedbl])}} \# attrs \\
\mycolX{48mm}{\code{xlabel!("My x")}} \# alternat. \\
\mycolX{25mm}{\code{clibrary(<clib>)}} \# import color library $\in$: \\ 
\textit{:Plots, :cmocean, :misc, :colorcet, :colorbrewer} \\
\mycolX{30mm}{\code{plotattr()}} \# query params \\

\textit{Plot-level attributes (more \href{https://docs.juliaplots.org/latest/generated/attributes\_plot/}{here}):}\\
{\scriptsize
\begin{tabular}{l l l}
    bg (color)      & size          & dpi \\
    fontfamily      & title         & legend \\
    framestyle      & aspect\_ratio & camera \\
    palette         \\
\end{tabular}
}\\
\textit{Grid attributes (more \href{https://docs.juliaplots.org/latest/generated/attributes\_plot/}{here}):}\\
{\scriptsize
\begin{tabular}{l l l}
    grid            & gridlinewidth         & link \\
    {[x|y|z]}lims     & [x|y|z]ticks          & [x|y|z]scale\\
    {[x|y]}guide      & [x|y]label \\
\end{tabular}
}

\textit{Series-level attributes (more \href{https://docs.juliaplots.org/latest/generated/attributes\_series/}{here}):}\\
{\scriptsize
\begin{tabular}{l l l}
    Points                  & Lines          & Surfaces \\ \hline
    markercolor             & linecolor     & fillrange \\
    markeralpha             & linealpha     & fillcolor \\
    markersize              & linestyle     & fillalpha \\
    markershape             & linewidth     & \\
    markerstroke-           & \\
    \phantom{xxxx}-color    & \\
    \phantom{xxxx}-alpha    & \\
    \phantom{xxxx}-width    & \\
\end{tabular}
}

%%%%%%%%%%%%%%%%%%%%%%%%%%%%%%%%%%%%%%%%%%
\subsection*{Plot Types}
\mycolX{45mm}{\code{plot!(p,seriestype = :scatter)}} \# series \\
\textit{Where seriestype $\in$:}\\
{\scriptsize
\begin{tabular}{l l l}
    line        & path          & steppre \\
    steppost    & sticks        & scatter \\
    heatmap     & hexbin        & barbins \\
    barhist     & stephist      & bins2d \\
    histogram2d & histogram3d   & density \\
    bar         & hline         & vline \\
    contour     & pie           & shape \\
    image       & path3d        & scatter3d \\
    surface     & wireframe     & contour3d \\
    volume  \\
\end{tabular}
}


%%%%%%%%%%%%%%%%%%%%%%%%%%%%%%%%%%%%%%%%%%
\subsection*{Screen Layout}
\mycolX{43mm}{\code{plot(x, y, layout = (4, 1))}} \# 4x1 vertically \\
\mycolX{43mm}{\code{plot(p1,p2,p3,p4,layout=(2,2))}} \# saved p\textquotesingle s\\
\mycolX{43mm}{\code{l = @layout [a\{0.6h\} b\{0.6w\} c]}} \# advanced \\
\mycolX{43mm}{\code{plot(x, y, layout = l)}} \# use above \\
\mycolX{35mm}{\code{BB = (x1,x2,y1,y2)}} \# set boundingbox \\
\mycolX{35mm}{\code{plot(x,y,inset=(1,BB))}} \# insetting \\


%%%%%%%%%%%%%%%%%%%%%%%%%%%%%%%%%%%%%%%%%%
\subsection*{Exporting \& Importing}
\textit{Save to .eps, .html, .pdf, .png, .ps, .svg, .tex, .text:}\\
\mycolX{35mm}{\code{savefig(\textquotedbl myplot.png\textquotedbl)}} \# from screen \\
\mycolX{35mm}{\code{savefig(p, \textquotedbl myplot.pdf\textquotedbl)}} \# from var p \\
\mycolX{35mm}{\code{png(fn)}} \# shorthand save as \\
\mycolX{35mm}{\code{img = load(\textquotedbl a.png\textquotedbl)}} \# load image \\
\mycolX{35mm}{\code{plot(x,y,img)}} \# plot an image \\


%%%%%%%%%%%%%%%%%%%%%%%%%%%%%%%%%%%%%%%%%%
\subsection*{Animations}
\textit{See \href{https://docs.juliaplots.org/latest/\#simple-is-beautiful-1}{here} for more examples.}\\
\mycolX{35mm}{\code{p = plot([sin, cos], zeros(0), leg = false)}} \\
\mycolX{35mm}{\code{anim = Animation()}} \\
\mycolX{35mm}{\code{for x = range(0, stop = 10$\pi$, length = 100)}}  \\
\phantom{xxx}\mycolX{35mm}{\code{push!(p, x, Float64[sin(x), cos(x)])}}   \\
\phantom{xxx}\mycolX{35mm}{\code{frame(anim)}} \\
\mycolX{35mm}{\code{end}} \\




%%%%%%%%%%%%%%%%%%%%%%%%%%%%%%%%%%%%%%%%%%
\subsection*{Extensions}
\textit{Use or create \href{http://docs.juliaplots.org/latest/recipes/\#recipes-1}{recipes} for often-generated plot-types. Eg, \href{https://github.com/JuliaPlots/StatsPlots.jl}{StatsPlot} allows visualization of data frames, distributions, boxplots, etc. Also see \href{https://github.com/JuliaPlots/GraphRecipes.jl}{GraphRecipes} for help plotting graphs. Alternatively, browse the \say{\href{http://docs.juliaplots.org/latest/ecosystem/\#ecosystem-1}{ecosystem}}.}


\mycolX{35mm}{\code{@df iris scatter( }} \# using Dataframes\\
\mycolX{35mm}{\code{    :SepalLength, :SepalWidth)}} \\ 
\mycolX{35mm}{\code{plot(Normal(3, 5))}} \# using Distributions  \\

%\mycolX{35mm}{\code{  }} \#  \\
